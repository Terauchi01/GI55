\section{まとめ}
本研究では,ミニ2048を用いて,ミニ2048におけるNタプルネットワークのタプルサイズおよびOptimistic Initialization(OI)の初期値が,
プレイヤの学習性能および探索によるスコアに与える影響について詳細に分析した
研究の実施にあたって3つのリサーチクエスチョン(RQ)を設定し,実験を行った.

\begin{itemize}
\item \textbf{RQ1} Nタプルネットワークにおけるタプルの大きさ,数,パラメータ数のスコアへの影響
\item \textbf{RQ2} Optimistic Initialization の初期値によるNタプルネットワークの学習への影響
\item \textbf{RQ3} Nタプルネットワーク評価関数とExpectimax探索の組合せにおける性能向上の依存関係
\end{itemize}

実験の結果,以下の重要な知見が得られた
\begin{itemize}
\item \textbf{RQ1について:}
パラメータ数の対数とスコアの関係は放物線的であり,タプルサイズ5から6付近で性能が最大となることが確認された.
これは,2048で広く用いられている6タプルによる評価関数の妥当性を裏付ける結果となった.

\item \textbf{RQ2について:}
OIの初期値が0の場合は局所最適解への収束やスコアのばらつきが大きく,学習の安定性に影響を与えることが確認された.
一方,初期値が大きすぎる場合は学習が停滞する可能性も示された.
適切な初期値(1200程度)を設定することで,より多くのパラメータを効果的に活用できることが示された.

\item \textbf{RQ3について:}
Expectimax探索による性能向上は,評価関数のパラメータ数やOIの初期値に依らず一貫して効果的であることが確認された.
探索深さ6では,どの評価関数でもパーフェクトプレイとの差が約半分に縮まることが示された.
\end{itemize}

今後の課題としては,これらの知見を2048に適用し,より大きなタプルサイズでの性能評価や,
Multistagingによってパラメータ数を変化させた場合の学習性能を評価することが挙げられる.
また本研究で得られた知見に基づき,2048プレイヤを実装することでより,高性能なプレイヤの実現が期待できる.


% 本研究では,ミニ2048におけるNタプルネットワークのタプルサイズおよびOptimistic Initialization(OI)の初期値が,
% プレイヤの学習性能および探索によるスコアに与える影響について詳細に分析した.
% 実験の結果,以下の重要な知見が得られた:
% \begin{itemize}
% \item パラメータ数の増加とスコアの関係は方物線的であり,ミニ2048では5か6を頂点とすることが確認された.
% \item OIの初期値は0ではスコアのばらつきが大きく,局所最適解にハマるのか,学習の安定性に影響を与えることが確認された.
% しかし大きな値を設定しすぎると学習が終わらない可能性も示された.
% \item OIの初期値を適切に設定すると,スコアの向上が続くパラメータ数が増加する傾向が確認できた.
% \item 探索を組み合わせることでどのパラメータ数でもスコアが向上することが確認された.
% \item 探索を実装した場合でもOIの値と平均スコアのばらつきの関係は変わらなかった.
% \end{itemize}

% 以上の結果から
% ミニ2048においてパラメータ数とNタプルネットワークのサイズがスコアに与える影響とOptimistic Initialization(OI)の初期値がスコアに与える影響を明らかにした.
% 今後の課題としては,これらの結果を踏まえて,2048で1から9までのタプルを用いた場合の学習性能を評価することと,
% Multistagingによってパラメータ数を変化させた場合の学習性能を評価することが挙げられる.
% それによって得られたタプルをExpectimax探索に組み込むことで,より高いスコアを達成することが挙げられる.




% % 5,6,7+別で8 1200
% 横軸にパラメータ数,縦軸にスコアのやつを入れる
% 2048で8タプルか9タプルのラインでいいんじゃないか?
% 全てのタプルのOI1200の平均点
% タプルの表で横にずらしたやつは並べる
% 増やす時は対応してると嬉しい,追加分は右に
% F (Full) vs M (Manual)
% Expectimax探索の深さ6と学習のスコアの曲線が欲しい
% 詳細な比較をタプル数の違いで比較する