\newpage
\section{まとめ}
本研究では、ミニ2048におけるNタプルネットワークのタプルサイズおよびOptimistic Initialization(OI)の初期値が、
プレイヤの学習性能および探索によるスコアに与える影響について詳細に分析した。
実験の結果、以下の重要な知見が得られた:
\begin{itemize}
\item パラメータ数の増加とスコアの関係は方物線的であり、ミニ2048ではNT5かNT6を頂点とすることが確認された。
\item OIの初期値は0ではスコアのばらつきが大きく、局所最適解にハマるのか、学習の安定性に影響を与えることが確認された。
しかし大きな値を設定しすぎると学習が終わらない可能性も示された。
\item OIの初期値を適切に設定すると、スコアの向上が続くパラメータ数が増加する傾向が確認できた。
\item 探索を組み合わせることでどのパラメータ数でもスコアが向上することが確認された。
\item 探索を実装した場合でもOIの値と平均スコアのばらつきの関係は変わらなかった。
\end{itemize}

以上の結果から
ミニ2048においてパラメータ数とNタプルネットワークのサイズがスコアに与える影響とOptimistic Initialization(OI)の初期値がスコアに与える影響を明らかにした。
今後の課題としては、これらの結果を踏まえて、2048でNT1からNT9までのタプルを用いた場合の学習性能を評価することと、
Multistagingによってパラメータ数を変化させた場合の学習性能を評価することが挙げられる。
それによって得られたタプルをExpectimax探索に組み込むことで、より高いスコアを達成することが挙げられる。
% NT5,6,7+別で8 1200
% 横軸にパラメータ数、縦軸にスコアのやつを入れる
% 2048で8タプルか9タプルのラインでいいんじゃないか?
% 全てのタプルのOI1200の平均点
% タプルの表で横にずらしたやつは並べる
% 増やす時は対応してると嬉しい,追加分は右に
% NTF (Full) vs NTM (Manual)
% Expectimax探索の深さ6と学習のスコアの曲線が欲しい
% 詳細な比較をタプル数の違いで比較する