\newpage
\section{まとめ}
% 本研究では,2048のコンピュータプレイヤを構築する有力な手法である山下らの研究~\cite{YaKN22J}を基に,完全解析を行った.
% その結果を活用し,Nタプルネットワーク単体,Nタプルネットワーク+Expectimax探索,Nタプルネットワーク+モンテカルロ木探索(MCTS)を組み合わせたプレイヤの挙動を分析した.
% プレイヤの特性を詳細に調査するため,完全解析が可能な同種のゲームミニ2048を用いて評価を行った.
% 実験の結果,Nタプルネットワークは適切に学習できていることが確認された.また,Nタプルネットワーク+Expectimax探索の組み合わせでは,
% 探索の深さを増やすことで得点が向上することが分かった.一方で,探索を深くすることで誤った選択をする場面が一部存在することも明らかになった.
% Nタプルネットワーク+MCTSの組み合わせに関しては,バックアップ方式としてE-maxが有効に機能することが確認された.
% さらに,Expectimax探索とMCTSの比較では,同程度のスコアを持つ場合,Expectimax探索は序盤の安定性に優れ,MCTSは中盤以降でより強みを発揮することが示された.
% 以上の結果から,本研究は2048のプレイヤ設計における探索手法の適用方法に関して新たな知見を提供した.
% 今後の課題として,MCTSのさらなる最適化やExpectimax探索の改良手法の検証,Nタプルネットワークのタプル設計の見直しが挙げられる.
本研究では、ミニ2048におけるNタプルネットワークのタプルサイズおよびOptimistic Initialization(OI)の初期値が、
プレイヤの学習性能および探索によるスコアに与える影響について詳細に分析した。
実験の結果、以下の重要な知見が得られた:
\begin{itemize}
\item タプルの数やサイズを適切に増やすことでプレイヤの性能は向上するが、
NT6を超えるとその効果は限定的となり、過剰なパラメータ数は汎化性能の低下を招く可能性がある。
\item OIの初期値は、学習初期における探索の幅を調整する上で有効に機能し、
特にOI=1200は多くの構成において高いスコアを示した。過大な初期値(OI=5400)やゼロに設定するよりも、中間的な値が最適である可能性が示唆された。
\item タプルの組み合わせ(TN)の違いによっても性能差が生じ、
一部の構成では学習済み評価関数が特定の局面において不安定な判断を行っていた。より良いTNの選定によって、盤面評価の精度が改善され、不要なミスの低減につながることが確認された。
\end{itemize}
% これらの結果は、2048のような評価関数に依存するゲームにおいて、
% 学習性能の最大化には単なるタプルサイズの増加だけでなく、適切なタプル形状の設計と初期値設定が不可欠であることを示している。