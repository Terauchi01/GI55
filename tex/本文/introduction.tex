\section{はじめに}
% \color{red}あとで修正する\color{black}
% 2048は確率的一人用パズルゲームである.
% 2048はルールは単純だが,マスターするのは難しいとされ,多くの研究の対象となってきた.
% 高得点を獲得できるようになったコンピュータプレイヤも多く存在する.
% これらの多くはTD学習とExpectimax探索で学習されたN-tupleネットワークに基づいて開発された.
「2048」はG. Cirulliによって作られた確率的一人ゲームである~\cite{2048}.
これまでに,2048のさまざまなコンピュータプレイヤが作られてきた.
現在最も成功しているアプローチは,強化学習によってチューニングしたNタプルネットワーク評価関数~\cite{SzJa14}とExpectimax探索~\cite{YWHC16}を組み合わせるものである.
その後,NタプルネットワークとExpectimax探索の組合せを基礎として,一般的もしくはゲームに特化した改良手法が多数提案されてきた~\cite{YWHC16,Mats17,Jask17,GuCW22}.
Gueiによって作られた最先端プレイヤは~\cite{GuCW22}ではExpectimax探索の深さが 6 の場合に平均得点 625\,377 を達成した.
著者らの研究の大きな目標は,それらのAI技術のそれぞれについて,どのように働いているかをより詳細に分析し説明することである.
本研究では、ミニ2048におけるNタプルネットワークのタプルサイズおよびOptimistic Initialization(OI)の初期値が、
プレイヤの性能および探索によるスコアに与える影響について実験的に評価する.
% モンテカルロ木探索(MCTS)は囲碁を始めとするさまざまなゲームで良い結果を残して来た探索手法である.しかし2048においては,watanabeとMatsuzakiによるプレイヤ~\cite{WaMa22}が平均得点533\,542,
% AlphaZeroによるプレイヤ~\cite{ASOH22}が平均得点約55万点であり
% Expectimax探索と比べて良いスコアを残せていない.

% これまでの研究~\cite{SzJa14,YWHC16,GuCW22,Mats17,Jask17}でも,各手法を適用する前後で平均得点や目標の達成割合の指標がどのように変化したかを議論されている.著者らは,そのような大局的な指標だけでなく,個別の局面においてどのようにプレイが変化するかのような詳細な分析を行いたいと考えている.

% しかしながら,「2048」は局面数が非常に大きいため,そのような詳細な分析を行うことは難しい.そこで著者らは,盤面サイズが$3\times3$である「ミニ2048」を対象として研究を行う.「ミニ2048」は完全解析(強解決,Strongly Solving)が可能である~\cite{YaKN22J}.
% 著者らは先行研究~\cite{TeKM23}において,tile downgrading~\cite{Jask17,GuCW22}の効果や,optimistic initializationを用いて学習したNタプルネットワーク評価関数の動作などについて詳細な分析を行った.
% 本研究では,先行研究\cite{Terauchi24}を発展させ,Nタプルネットワーク評価関数とMCTSを組み合わせたプレイヤの動作について詳細に分析を行う.
% 本研究では山下らの研究~\cite{YaKN22J}を基に完全解析を行い,その結果を用いてNタプルネットワークとNタプルネットワーク+Expectimax探索とNタプルネットワーク+モンテカルロ木探索を組み合わせたプレイヤの動作について詳細に分析を行った.
% \cite{TeKM23}\cite{Terauchi24}\cite{Terauchi25}

% 本研究の貢献は以下の4点である.
% \begin{itemize}
%     % \item パーフェクトプレイヤを用いて,Nタプルネットワークプレイヤの分析を行った.
%     % \item Nタプルネットワークに使用されている既存手法Multistaging~\cite{YWHC16}, TC学習~\cite{Jask17}, Optimistic initialization~\cite{GuCW22}の分析を行った.
%     \item Expectimax探索とNタプルネットワークの組み合わせについて,探索の深さによる得点の変化を分析した.
%     \item Expectimax探索とNタプルネットワークの組み合わせについて,完全解析によって得られたゲーム値データベースと比較することにより,最善手を選択した割合や最善手との評価値の差などを分析した.
%     \item モンテカルロ木探索とNタプルネットワークの組み合わせについて,各パラメータの変化とその影響について詳細に分析を行った.
%     \item モンテカルロ木探索とNタプルネットワークの組み合わせについて,新しいバックアップ手法の提案とその効果について分析を行った.
%     % \item モンテカルロ木探索とNタプルネットワークの組み合わせについて,シミュレーション回数と得点の関係について分析を行った.
%     \item Nタプルネットワークに対してExpectimax探索とモンテカルロ木探索をそれぞれ組み合わせた場合の比較と分析を行った.
% \end{itemize}


% 本研究では2048のコンピュータプレイヤの探索方法の1つである,MCTSについてミニ2048の完全解決の結果とExpectimax探索の比較を用いて,解析に取り組んだ.
% 全体としてはMCTSはNT4との組み合わせではExpectimax探索には及ばないものの得点向上が見込める,という妥当な結果を得た.
% UCBのC値を子ノードの最大にする手法は平均得点が高くなることが分かった.
% しかしMCTSをExpectimax探索に近づけるような期待値を使った選択方式の平均得点が下がったのは意外な結果であった.
% またと期待値を使った選択方式の場合C値がほとんど影響しないことが分かった.
% 今後の課題は,MCTSの選択を行う際の関数で別の種類のものを試すことと,シミュレーション回数を増やした際にあまり平均点向上に結びつかない理由を解明することである.
% Expectimaxでは平均5200点程度まで達成できている為,MCTSにも平均点向上の余地は残されていると考える.