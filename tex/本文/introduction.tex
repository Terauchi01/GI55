\section{はじめに}
「2048」はG.~Cirulliによって作られた確率的一人ゲームであり~\cite{2048}.
これまでにさまざまなコンピュータプレイヤが作られてきた.
現在最も成功しているアプローチは,強化学習によってチューニングしたNタプルネットワーク評価関数~\cite{SzJa14}とExpectimax探索~\cite{YWHC16}を組み合わせるものである.
その後,NタプルネットワークとExpectimax探索の組合せを基礎として,一般的もしくはゲームに特化した改良手法が多数提案されてきた~\cite{YWHC16,Mats17,Jask17,GuCW22}.
Gueiによって作られた最先端プレイヤ~\cite{GuCW22}では,大きさ6のタプル8個からなるネットワーク2つを切り替えて用いる評価関数,より幅広く学習するためのOptimistic Initialization,ゲーム特化型の改良手法であるTile downgrading,およびExpectimax探索を組み合わせることで,平均得点 625\,377 を達成した.

Nタプルネットワークやニューラルネットワークでは,一般に,ネットワーク内のパラメータ数が増えるほど性能が向上すると言われている~\cite{強化学習の教科書か,LLMの文脈で何かあるか?}.一方で,パラメータ数が増えすぎると,学習に必要となるデータが莫大になるという問題に加えて,過学習(過適合)の問題も発生する.

2048のコンピュータプレイヤにおいては,これまでネットワークのパラメータ数を増やすことで性能向上が図られてきた.
2048にNタプルネットワークを用いる最初の研究である Szubert と Ja\'{s}kowski による研究~\cite{SzJa14}では,まず大きさ4のタプル17個からなるネットワーク(パラメータ数$1.11\times 10^6$)が用いられ,次に6タプル2個と4タプル2個からなるネットワーク(パラメータ数$3.37\times 10^7$)が用いられていた.
続く Wuらによる研究~\cite{W???14}では,ネットワークが大きさ6のタプル4つからなるもの(パラメータ数$6.71\times 10^7$)へ拡張された.
2018年時点での最先端プレイヤ~\cite{Jask18}では,6タプル5個にredundant encodingを組み合わせたネットワーク(パラメータ数 $8.42\times 10^7$)16個をゲーム進行に合わせて切り替える手法がとられている.論文投稿時点のGuei による最先端プレイヤ~\cite{GuCW22}では,Matsuzak~\cite{Mats16}が実験的に求めた6タプル8つの組合せ(パラメータ数 $1.34\times 10^8$)2つを切り替えて用いている.このGuei によるプレイヤの学習では,Optimistic Initialization を用いてより幅広く学習する工夫が取り入れられている.
一方,ニューラルネットワークによる評価関数を用いるプレイヤでは,Matsuzaki~\cite{Mats19}が,畳み込みネットワークのパラメータ数を $1.85\times 10^5$ から $2.90\times 10^6$ まで変化させたときに性能が向上することを報告している.

ここで生じる疑問は,2048のNタプル評価関数において,性能向上が見られる範囲でどこまでパラメータ数を増やしていけるのか,という点である.Oka と Matsuzaki~\cite{OkMa},および,Matsuzaki~\cite{Mats16} は,大きさ6のタプルだけでなく,大きさ7のタプル複数個を系統的に組み合わせる手法を示し,同一条件での比較においては大きさ6のネットワークよりも大きさ7 のネットワークのほうが性能が高くなりうることを示した.
しかしながら,大きさ8のタプルからなるネットワークでは,単純に実装するとパラメータ数が $4.29\times 10^{12}$ と巨大になり,メモリサイズおよび学習コストの観点から,現時点では実現は困難である.


そこで本研究では,大きさが小さくパラメータ数を減らすことができ,また,すでに完全解析による真の正解が分かってるミニ2048を用いて,NタプルネットワークのタプルサイズとOptimistic Initialization(OI)の初期値がプレイヤの性能に与える影響について実験的に評価する.本研究の実施にあたって設定したリサーチクエスチョンは次の3つである.



\vspace{0.5em}
\noindent 本研究の主な貢献は以下の4点である:
\begin{itemize}
\item Nタプルネットワークのタプルサイズおよび数(=パラメータ数)が,スコアに与える影響を定量的に評価した.
\item Optimistic Initialization(OI)の初期値が,学習の安定性とスコアに与える影響を評価した.
\item 探索(Expectimax)による強化が,Nタプルの構成およびOI設定とどう相互作用するかを明らかにした.
\item タプル数やOIによる性能変化を可視化・解析し,設計指針の定量的な知見を提供した.
\end{itemize}

実験の結果,以下の知見が得られた.
\begin{itemize}
\item パラメータ数とスコアの関係は方物線的であり,NT5~NT6付近が性能のピークであった.
\item OIの初期値が小さすぎると学習が不安定になり,大きすぎると学習が進まない傾向があった.
\item 適切なOIの設定により,有効な学習が進み,パラメータ数が増えてもスコアが向上する構成が見られた.
\item Expectimax探索はスコア全体を底上げするが,OIの影響は限定的であった.
\end{itemize}

以上より,Nタプルネットワークの構成や初期化戦略が2048型ゲームにおけるAI性能に大きく影響することを示し,
今後の高性能プレイヤ設計に向けた基礎的な知見を提供する.