\section{はじめに}
「2048」はG. Cirulliによって作られた確率的一人ゲームである~\cite{2048}.
これまでに,2048のさまざまなコンピュータプレイヤが作られてきた.
現在最も成功しているアプローチは,強化学習によってチューニングしたNタプルネットワーク評価関数~\cite{SzJa14}とExpectimax探索~\cite{YWHC16}を組み合わせるものである.
その後,NタプルネットワークとExpectimax探索の組合せを基礎として,一般的もしくはゲームに特化した改良手法が多数提案されてきた~\cite{YWHC16,Mats17,Jask17,GuCW22}.
Gueiによって作られた最先端プレイヤは~\cite{GuCW22}ではExpectimax探索の深さが 6 の場合に平均得点 625\,377 を達成した.
著者らの研究の大きな目標は,それらのAI技術のそれぞれについて,どのように働いているかをより詳細に分析し説明することである.
本研究では、ミニ2048におけるNタプルネットワークのタプルサイズおよびOptimistic Initialization(OI)の初期値が、
プレイヤの性能および探索によるスコアに与える影響について実験的に評価した.

\vspace{0.5em}
\noindent 本研究の主な貢献は以下の4点である:
\begin{itemize}
\item Nタプルネットワークのタプルサイズおよび数(=パラメータ数)が、スコアに与える影響を定量的に評価した.
\item Optimistic Initialization(OI)の初期値が、学習の安定性とスコアに与える影響を評価した.
\item 探索(Expectimax)による強化が,Nタプルの構成およびOI設定とどう相互作用するかを明らかにした.
\item タプル数やOIによる性能変化を可視化・解析し,設計指針の定量的な知見を提供した.
\end{itemize}

実験の結果,以下の知見が得られた.
\begin{itemize}
\item パラメータ数とスコアの関係は方物線的であり,NT5~NT6付近が性能のピークであった.
\item OIの初期値が小さすぎると学習が不安定になり,大きすぎると学習が進まない傾向があった.
\item 適切なOIの設定により,有効な学習が進み,パラメータ数が増えてもスコアが向上する構成が見られた.
\item Expectimax探索はスコア全体を底上げするが,OIの影響はGreedyプレイ時と同様に限定的であった.
\end{itemize}

以上より,Nタプルネットワークの構成や初期化戦略が2048型ゲームにおけるAI性能に大きく影響することを示し,
今後の高性能プレイヤ設計に向けた基礎的な知見を提供する.