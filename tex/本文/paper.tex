%%
%% 研究報告用スイッチ
%% [techrep]
%%
%% 欧文表記無しのスイッチ(etitle,eabstractは任意)
%% [noauthor]
%%

\documentclass[techrep,submit,noauthor,platex,dvipdfmx]{ipsj}

% フォントと日本語関連パッケージ
% \usepackage[deluxe,uplatex]{otf}
% \usepackage[T1]{fontenc}

% グラフィックス関連
\usepackage[dvipdfmx]{graphicx} 
\usepackage{subcaption}

% 数式関連
\usepackage{latexsym}
\usepackage{amsmath}
\usepackage{amstext}
\usepackage{amssymb}
\usepackage{multirow}

% その他のパッケージ
\usepackage{url}
\makeatletter
\def\@footyear{2025}
\makeatother

% hyperrefは最後に読み込む
% \usepackage[dvipdfmx,setpagesize=false]{hyperref}

% \def\Underline{\setbox0\hbox\bgroup\let\\\endUnderline}
% \def\endUnderline{\vphantom{y}\egroup\smash{\underline{\box0}}\\}
% \def\|{\verb|}

% \setcounter{巻数}{59}
% \setcounter{号数}{1}
% \setcounter{page}{1}


% \受付{2016}{3}{4}
% \再受付{2015}{7}{16}   %省略可能
% \再再受付{2015}{7}{20} %省略可能
% \再再受付{2015}{11}{20} %省略可能
% \採録{2016}{8}{1}




\begin{document}


\title{Nタプルネットワークの大きさと学習性能の関係:\\ミニ2048を用いた実験・評価}

% \etitle{}

% \affiliate{IPSJ}{情報処理学会\\
% IPSJ, Chiyoda, Tokyo 101--0062, Japan}


\affiliate{KUTGE}{高知工科大学大学院工学研究科}
\affiliate{KUTSI}{高知工科大学情報学群}

% \paffiliate{JU}{高知工科大学\\
% Kochi University of Technology\\}

\author{寺内 俊輔}{Shunsuke Terauchi}{KUTGE}[295141a@gs.kochi-tech.ac.jp]
\author{松崎 公紀}{Kimnori Matsuzaki}{KUTSI}[matsuzaki.kiminori@kochi-tech.ac.jp]

\begin{abstract}
    2048では,NタプルネットワークとTD学習を拡張した手法及びExpectimax探索の組み合わせにより優れたプレイヤが作られている.とくに,これまでNタプルネットワークのパラメータ数を増やすことで性能向上が図られてきた.しかしながら,その傾向がどこまで続くかについては,議論がなされていないのが現状である.
    本研究では,ミニ2048において利用可能な1タプルから9タプルまでの各タプル全てからなる組み合わせと,それぞれにおける妥当な組合せを用いて,Nタプルネットワークの大きさと学習性能の関係を実験的に評価する.
    具体的には,Nタプルネットワークのパラメータ数とスコアの関係,Optimistic Initialization 手法の初期値と学習への影響,Expectimax探索を組み合わせたときの性能向上について詳しく評価した.
\end{abstract}

\begin{jkeyword}
2048, Nタプルネットワーク, Expectimax探索, ミニ2048, 強化学習
\end{jkeyword}

% \begin{eabstract}
% This document is a guide to prepare a draft for submitting to IPSJ
% Journal, and the final camera-ready manuscript of a paper to appear in
% IPSJ Journal, using {\LaTeX} and special style files.  Since this
% document itself is produced with the style files, it will help you to
% refer its source file which is distributed with the style files.
% \end{eabstract}

% \begin{ekeyword}
% IPSJ Journal, \LaTeX, style files, ``Dos and Don'ts'' list
% \end{ekeyword}
\maketitle
\section{はじめに}
% \color{red}あとで修正する\color{black}
% 2048は確率的一人用パズルゲームである.
% 2048はルールは単純だが,マスターするのは難しいとされ,多くの研究の対象となってきた.
% 高得点を獲得できるようになったコンピュータプレイヤも多く存在する.
% これらの多くはTD学習とExpectimax探索で学習されたN-tupleネットワークに基づいて開発された.
「2048」はG. Cirulliによって作られた確率的一人ゲームである~\cite{2048}.
これまでに,2048のさまざまなコンピュータプレイヤが作られてきた.
現在最も成功しているアプローチは,強化学習によってチューニングしたNタプルネットワーク評価関数~\cite{SzJa14}とExpectimax探索~\cite{YWHC16}を組み合わせるものである.
その後,NタプルネットワークとExpectimax探索の組合せを基礎として,一般的もしくはゲームに特化した改良手法が多数提案されてきた~\cite{YWHC16,Mats17,Jask17,GuCW22}.
Gueiによって作られた最先端プレイヤは~\cite{GuCW22}ではExpectimax探索の深さが 6 の場合に平均得点 625\,377 を達成した.
著者らの研究の大きな目標は,それらのAI技術のそれぞれについて,どのように働いているかをより詳細に分析し説明することである.
本研究では、ミニ2048におけるNタプルネットワークのタプルサイズおよびOptimistic Initialization(OI)の初期値が、
プレイヤの性能および探索によるスコアに与える影響について実験的に評価する.
% モンテカルロ木探索(MCTS)は囲碁を始めとするさまざまなゲームで良い結果を残して来た探索手法である.しかし2048においては,watanabeとMatsuzakiによるプレイヤ~\cite{WaMa22}が平均得点533\,542,
% AlphaZeroによるプレイヤ~\cite{ASOH22}が平均得点約55万点であり
% Expectimax探索と比べて良いスコアを残せていない.

% これまでの研究~\cite{SzJa14,YWHC16,GuCW22,Mats17,Jask17}でも,各手法を適用する前後で平均得点や目標の達成割合の指標がどのように変化したかを議論されている.著者らは,そのような大局的な指標だけでなく,個別の局面においてどのようにプレイが変化するかのような詳細な分析を行いたいと考えている.

% しかしながら,「2048」は局面数が非常に大きいため,そのような詳細な分析を行うことは難しい.そこで著者らは,盤面サイズが$3\times3$である「ミニ2048」を対象として研究を行う.「ミニ2048」は完全解析(強解決,Strongly Solving)が可能である~\cite{YaKN22J}.
% 著者らは先行研究~\cite{TeKM23}において,tile downgrading~\cite{Jask17,GuCW22}の効果や,optimistic initializationを用いて学習したNタプルネットワーク評価関数の動作などについて詳細な分析を行った.
% 本研究では,先行研究\cite{Terauchi24}を発展させ,Nタプルネットワーク評価関数とMCTSを組み合わせたプレイヤの動作について詳細に分析を行う.
% 本研究では山下らの研究~\cite{YaKN22J}を基に完全解析を行い,その結果を用いてNタプルネットワークとNタプルネットワーク+Expectimax探索とNタプルネットワーク+モンテカルロ木探索を組み合わせたプレイヤの動作について詳細に分析を行った.
% \cite{TeKM23}\cite{Terauchi24}\cite{Terauchi25}

% 本研究の貢献は以下の4点である.
% \begin{itemize}
%     % \item パーフェクトプレイヤを用いて,Nタプルネットワークプレイヤの分析を行った.
%     % \item Nタプルネットワークに使用されている既存手法Multistaging~\cite{YWHC16}, TC学習~\cite{Jask17}, Optimistic initialization~\cite{GuCW22}の分析を行った.
%     \item Expectimax探索とNタプルネットワークの組み合わせについて,探索の深さによる得点の変化を分析した.
%     \item Expectimax探索とNタプルネットワークの組み合わせについて,完全解析によって得られたゲーム値データベースと比較することにより,最善手を選択した割合や最善手との評価値の差などを分析した.
%     \item モンテカルロ木探索とNタプルネットワークの組み合わせについて,各パラメータの変化とその影響について詳細に分析を行った.
%     \item モンテカルロ木探索とNタプルネットワークの組み合わせについて,新しいバックアップ手法の提案とその効果について分析を行った.
%     % \item モンテカルロ木探索とNタプルネットワークの組み合わせについて,シミュレーション回数と得点の関係について分析を行った.
%     \item Nタプルネットワークに対してExpectimax探索とモンテカルロ木探索をそれぞれ組み合わせた場合の比較と分析を行った.
% \end{itemize}


% 本研究では2048のコンピュータプレイヤの探索方法の1つである,MCTSについてミニ2048の完全解決の結果とExpectimax探索の比較を用いて,解析に取り組んだ.
% 全体としてはMCTSはNT4との組み合わせではExpectimax探索には及ばないものの得点向上が見込める,という妥当な結果を得た.
% UCBのC値を子ノードの最大にする手法は平均得点が高くなることが分かった.
% しかしMCTSをExpectimax探索に近づけるような期待値を使った選択方式の平均得点が下がったのは意外な結果であった.
% またと期待値を使った選択方式の場合C値がほとんど影響しないことが分かった.
% 今後の課題は,MCTSの選択を行う際の関数で別の種類のものを試すことと,シミュレーション回数を増やした際にあまり平均点向上に結びつかない理由を解明することである.
% Expectimaxでは平均5200点程度まで達成できている為,MCTSにも平均点向上の余地は残されていると考える.
\section{ミニ2048}
本研究はミニ2048~\cite{YaKN22J}を研究対象として使用する.
ミニ2048は,$3\times 3$ の盤面でプレイされることを除いて,確率的一人ゲーム{2048}と同じゲームである.

\subsection{ルール}

ミニ2048は,$3\times 3$ の盤面でプレイされる.
初期局面は9マスの内どこか2マスに2(確率0.9)か4(確率0.1)の数字タイルがランダムに置かれた盤面からなる.

各局面において,プレイヤは上下左右いずれかの方向を選択する.
すると各数字タイルはその方向にできるだけ移動する.
移動した結果,2つの同じ数字のタイルが移動方向に衝突するとこれらは合体してその合計値のタイルとなり,
その合計値がスコアに加算される.合体してできたタイルは,同じターンでは別のタイルと合体することはない.
例えば,盤面の行が \verb*|_2_|,\verb*|22_|,\verb*|422| であるとき,右を選択するとそれぞれ \verb*|__2|,\verb*|__4|,\verb*|_44| へと変化する.
その後,空白のマスのうちのランダムな1マスに2(確率0.9)か4(確率0.1)のタイルが置かれる.

プレイヤはいずれかのタイルが移動または衝突するような方向しか選択することができない.いずれの方向も選択できなくなるとゲームは終了する.
このゲームの目標はゲームが終了する前に出来るだけ高得点を獲得することである.

\subsection{用語の導入}
通常の2048と同様にミニ2048における1ターンは,「移動・合体ステップ」と「新規タイルステップ」の2ステップからなる.これらステップの前後の状態を区別するため,以下の用語を導入する.
\begin{figure}[t]
  \centering\includegraphics[width=.99\linewidth]{pdf/state_afterstate.drawio.pdf}
  \caption{state,afterstate,progressの例}
  \label{afterstate}
 \end{figure}
\begin{description}
  \item[state] プレイヤが手を選択する盤面状態(とスコア)を\emph{state}と呼ぶ.
  \item[afterstate] プレイヤが手を選択してタイルが移動・合体した直後の盤面状態(とスコア)を\emph{afterstate}と呼ぶ.すなわち,afterstateは新規タイルが出現する前の盤面状態である.
 \end{description}

(通常の2048同様に)ミニ2048では,新しく出現するタイルはランダムに 2 か 4 の値をとる.そのため,単純にターン数をゲームの流さや進行度の指標に用いるには不都合がある.
この問題を解決するため,本研究では以下の指標を用いる.
\begin{description}
  % (\emph{時刻}~\cite{YaKa23J})これ抜いた
 \item[progress] タイルの値の合計値の半分を\emph{progress}~\cite{TeKM23}と呼ぶ.progress は,1ターンで1(新規タイルが2の場合)または2(新規タイルが4の場合)だけ増加する.
\end{description}

図\ref{afterstate}は,初期局面から始まるゲームの流れにおいて,state,afterstate,progress について図示したものである.


\subsection{完全解析とその結果}

ミニ2048は確率的一人ゲームであり,その完全解析とは各状態に対して期待スコア求めることである.
ミニ2048は,到達可能な状態数が $10^9$ 以下と小さいため,現実的な時間で完全解析ができる.
山下ら~\cite{YaKN22J}は,ミニ2048の完全解析に最初に取り組み,そこでは幅優先探索による状態列挙と,列挙した状態を用いる後退解析を行った.
また,著者ら~\cite{TeKM23}も完全解析の追試を行い,深さ優先探索による後退解析で,結果の正しさを確認した.

完全解析の結果について,重要なものを以下に示す.
初期状態のいずれかから到達可能な state の数は 48\,713\,519,afterstate の数は 31\,431\,374 である.
初期状態の期待スコアは,5\,468.49 である.
各 afterstate に対する期待スコアを格納したものを valueDB と呼ぶ.

完全解析で得られる valueDB を用いると,各局面において最適な手を選択するパーフェクトプレイヤを実現できる.
ただし,ミニ2048は確率的一人ゲームのため,決定的に最善手を選択するパーフェクトプレイヤであっても,ゲームごとに
プレイの結果は異なることに注意が必要である.
図\ref{pp-play}は,パーフェクトプレイヤが1万ゲームを行った際の,progress ごとの生存率とゲーム終了時のスコアを示している.
表\ref{pp-specific}は,パーフェクトプレイヤが256,512,1024タイルに到達したときの進捗状況,スコア,生存率を示している.
パーフェクトプレイヤでも,512タイルに到達した後,1024タイルに到達する前は,生存率が急激に低下することが分かる.
図\ref{pp-play}より,生存率が急激に下がるタイミングがいくつかある.
本研究では,そのような生存率が下がる部分を\emph{難易度の高い領域}と呼ぶ.
\begin{table}[t]
  \caption{Progress, score, and alive ratio of perfect player}
  \label{pp-specific}
  \centering\begin{tabular}{lrrr}
   \hline\hline
   Condition & ~~Progress & ~~~~Score~ & ~~~~~~Alive~~ \\
   \hline
  256-tile                          & 136~~~ & 1\,750~ & 99.53\% \\
  512-tile                          & 263~~~ & 4\,000~ & 73.84\% \\
  512-tile \& 256-tile              & 391~~~ & 5\,750~ & 54.40\% \\
  512-tile \& 256-tile \& 128-tile  & 456~~~ & 6\,500~ & 40.49\% \\
  1024-tile                         & 511~~~ & 9\,000~ &  1.07\% \\
   \hline
\end{tabular}
\end{table}

\begin{figure}[t]
  \centering\includegraphics[width=\linewidth]{pdf/pp-play.pdf}
  \caption{パーフェクトプレイヤの生存率とスコア \cite{TeKM23}}
  \label{pp-play}
\end{figure}

% \section{パーフェクトプレイヤの結果を用いたミニ2048の分析}
% \subsection{プレイヤにノイズを加えた場合の平均得点の変化}

% 一般に,stateやafter stateの評価関数は完全に正確であるとは限らない.そこ,で評価関数が不正確な場合のスコアの変化を調査するため,valueDB にノイズを加えたシミュレーションを行った.
% ノイズは平均 0,分散 $\sigma^2$ の正規分布から生成し,標準偏差 $\sigma$ を 0 から 500 まで変化させた.
% このように変化させた valueDB を用いる貪欲プレイヤに 10,000 回のゲームをプレイさせ,その平均スコアを図~\ref{error_averagescore} に示す.

% % \begin{figure} 
% %   \centering\includegraphics[height=0.40\textheight]{pdf/error_averagescore.pdf} 
% %   \caption{ノイズを加えたプレイヤの平均スコア} 
% %   \label{error_averagescore} 
% % \end{figure}

% 図~\ref{error_averagescore} より,標準偏差 $\sigma$ を大きくすると,平均スコアは単調に減少することが確認できる.
% 例えば,平均スコアが 5000,4000,3000,2000 となる $\sigma$ の値は,それぞれ $\sigma \approx 25, 75, 145, 270$ であった.

\section{本研究で用いるプレイヤ}
\subsection{Nタプルネットワーク}
\label{sec:Ntuple}

% 2048における最も成功したプレイヤの多くは,Nタプルネットワークに基づく評価関数を強化学習によってチューニングするアプローチを採用している~\cite{SzJa14}.
% Gueiらの最新のプレイヤ~\cite{GuCW22}も,Matsuzaki~\cite{Mats16}が提案したタプルの組合せをベースに,Expectimax探索やMultistaging\cite{YWHC16},Optimistic Initialization,Tile Downgrading\cite{GuCW22}などの改良を加えることで高い性能を達成している.

本研究では,表\ref{tuples}に示すとおり,1タプルから9タプルまでのNタプルネットワークを合計で15種類設計して用いた.

まず,特徴抽出する $N$ マス(図では丸で示される)がすべて上下左右に連結しているようなものを,有効な $N$ タプルとした.
その結果,$N=1$ から $N=9$ まで順に有効な $N$ タプルが,3通り,2通り,5通り,6通り,9通り,8通り,7通り,3通り,1通り得られた.
指定した大きさ $N$ について,有効な $N$ タプルをすべて含むNタプルネットワークを $N$-Full と名付けた(本文および図表においては,より短く \textsf{1F} などと表記する).

2048におけるNタプルネットワークの設計では,有望そうな形をいくつか人手で作成し,それを平行移動させて得られる
タプルを組合せる手法がよく用いられている~\cite{SzJa14,YWHC16,Jask17}.
そこで,この考え方に基づくタプルの組合せを $N$-Manual として設計した(本文および図表においては,より短く \textsf{3M} などと表記する).
ただし,1タプル,2タプル,9タプルでは,有効な $N$ タプルがすべて有望そうな形をしていることから,\textsf{1M},\textsf{2M},\textsf{9M}はそれぞれ\textsf{1F},\textsf{2F},\textsf{9F}と同一である.

なお,Nタプルネットワークで評価値を計算する際には,ミニ2048の盤面の持つ対称性(回転・反転)を活用し,各タプルに対して8通りの位置からのサンプリングを行う.また,後述する Multistaging により各プレイヤはNタプルネットワークを2つ持つことから,表\ref{tuples}に示すパラメータ数は,タプルサイズを $N$ として
\[
 \mbox{パラメータ数} = 11^N \times 2
\]
により計算される値である.

\begin{table}[t]
  \caption{タプルサイズと組合せの一覧}
  \label{tuples}
  \centering\begin{tabular}{llr}
   \hline
   \hline
   名称 & \hspace{20pt}タプルの組合せ & パラメータ数\\
   \hline
   \raisebox{10pt}{\textsf{1F}}\raisebox{28pt}{~}
          & \includegraphics[height=22pt]{pdf/tuples/1tuple_6_page1.pdf}~
            \includegraphics[height=22pt]{pdf/tuples/1tuple_6_page2.pdf}~
            \includegraphics[height=22pt]{pdf/tuples/1tuple_6_page3.pdf} & \raisebox{10pt}{66}\\
   \hline
   \raisebox{10pt}{\textsf{2F}}\raisebox{28pt}{~}
          & \includegraphics[height=22pt]{pdf/tuples/2tuple_12_page1.pdf}~
            \includegraphics[height=22pt]{pdf/tuples/2tuple_12_page2.pdf}& \raisebox{10pt}{484}\\
   \hline
   \raisebox{10pt}{\textsf{3M}}\raisebox{28pt}{~}
          & \includegraphics[height=22pt]{pdf/tuples/3tuple_144_page1.pdf}~
            \includegraphics[height=22pt]{pdf/tuples/3tuple_144_page3.pdf}~
            \includegraphics[height=22pt]{pdf/tuples/3tuple_144_page2.pdf}& \raisebox{10pt}{7,986}\\
   \hline
   \raisebox{10pt}{\textsf{3F}}\raisebox{28pt}{~}
          & \includegraphics[height=22pt]{pdf/tuples/3tuple_2673_page1.pdf}~
            \includegraphics[height=22pt]{pdf/tuples/3tuple_2673_page5.pdf}~
            \includegraphics[height=22pt]{pdf/tuples/3tuple_2673_page2.pdf}~
            \includegraphics[height=22pt]{pdf/tuples/3tuple_2673_page4.pdf}~
            \includegraphics[height=22pt]{pdf/tuples/3tuple_2673_page3.pdf}& \raisebox{10pt}{13,310}\\
   \hline
   \raisebox{10pt}{\textsf{4M}}\raisebox{28pt}{~}
          & \includegraphics[height=22pt]{pdf/tuples/4tuple_301_page1.pdf}~
            \includegraphics[height=22pt]{pdf/tuples/4tuple_301_page3.pdf}~
            \includegraphics[height=22pt]{pdf/tuples/4tuple_301_page2.pdf}& \raisebox{10pt}{87,846}\\
   \hline
   % \raisebox{10pt}{\textsf{4F}}\raisebox{28pt}{~}
   \multirow{2}{*}{\textsf{4F}}\raisebox{28pt}{~}
          & \includegraphics[height=22pt]{pdf/tuples/4tuple_44755_page1.pdf}~
            \includegraphics[height=22pt]{pdf/tuples/4tuple_44755_page5.pdf}~
            \includegraphics[height=22pt]{pdf/tuples/4tuple_44755_page3.pdf}& \multirow{2}{*}{175,692}\\
          & \includegraphics[height=22pt]{pdf/tuples/4tuple_44755_page6.pdf}~
            \includegraphics[height=22pt]{pdf/tuples/4tuple_44755_page2.pdf}~
            \includegraphics[height=22pt]{pdf/tuples/4tuple_44755_page4.pdf}\\
   \hline
   \raisebox{10pt}{\textsf{5M}}\raisebox{28pt}{~}
          & \includegraphics[height=22pt]{pdf/tuples/5tuple_298_page1.pdf}~
            \includegraphics[height=22pt]{pdf/tuples/5tuple_298_page2.pdf}~
            \includegraphics[height=22pt]{pdf/tuples/5tuple_298_page3.pdf} & \raisebox{10pt}{966,306}\\
   \hline
   \multirow{2}{*}{\textsf{5F}}\raisebox{28pt}{~}
          & \includegraphics[height=22pt]{pdf/tuples/5tuple_896673_page1.pdf}~
            \includegraphics[height=22pt]{pdf/tuples/5tuple_896673_page3.pdf}~
            \includegraphics[height=22pt]{pdf/tuples/5tuple_896673_page4.pdf}~
            \includegraphics[height=22pt]{pdf/tuples/5tuple_896673_page2.pdf}~
            \includegraphics[height=22pt]{pdf/tuples/5tuple_896673_page5.pdf}& \multirow{2}{*}{2,898,918}\\
          & \includegraphics[height=22pt]{pdf/tuples/5tuple_896673_page6.pdf}~
            \includegraphics[height=22pt]{pdf/tuples/5tuple_896673_page7.pdf}~
            \includegraphics[height=22pt]{pdf/tuples/5tuple_896673_page8.pdf}~
            \includegraphics[height=22pt]{pdf/tuples/5tuple_896673_page9.pdf}\\
   \hline
   \raisebox{10pt}{\textsf{6M}}\raisebox{28pt}{~}
          & \includegraphics[height=22pt]{pdf/tuples/6tuple_16_page1.pdf}~
            \includegraphics[height=22pt]{pdf/tuples/6tuple_16_page2.pdf} & \raisebox{10pt}{7,086,244}\\
   \hline
   \multirow{2}{*}{\textsf{6F}}\raisebox{28pt}{~}
          & \includegraphics[height=22pt]{pdf/tuples/6tuple_26835_page1.pdf}~
            \includegraphics[height=22pt]{pdf/tuples/6tuple_26835_page2.pdf}~
            \includegraphics[height=22pt]{pdf/tuples/6tuple_26835_page3.pdf}~
            \includegraphics[height=22pt]{pdf/tuples/6tuple_26835_page4.pdf}& \multirow{2}{*}{28,344,976}\\
          & \includegraphics[height=22pt]{pdf/tuples/6tuple_26835_page5.pdf}~
            \includegraphics[height=22pt]{pdf/tuples/6tuple_26835_page6.pdf}~
            \includegraphics[height=22pt]{pdf/tuples/6tuple_26835_page7.pdf}~
            \includegraphics[height=22pt]{pdf/tuples/6tuple_26835_page8.pdf}\\
   \hline
   \raisebox{10pt}{\textsf{7M}}\raisebox{28pt}{~}
          & \includegraphics[height=22pt]{pdf/tuples/7tuple_0_page1.pdf} & \raisebox{10pt}{38,974,342}\\
   \hline
   \raisebox{10pt}{\textsf{7F}}\raisebox{28pt}{~}
          & \includegraphics[height=22pt]{pdf/tuples/7tuple_248_page1.pdf}~
            \includegraphics[height=22pt]{pdf/tuples/7tuple_248_page2.pdf}~
            \includegraphics[height=22pt]{pdf/tuples/7tuple_248_page3.pdf}~
            \includegraphics[height=22pt]{pdf/tuples/7tuple_248_page4.pdf}& \multirow{2}{*}{272,820,394}\\
          & \includegraphics[height=22pt]{pdf/tuples/7tuple_248_page5.pdf}~
            \includegraphics[height=22pt]{pdf/tuples/7tuple_248_page6.pdf}~
            \includegraphics[height=22pt]{pdf/tuples/7tuple_248_page7.pdf}\\
   \hline
   \raisebox{10pt}{\textsf{8M}}\raisebox{28pt}{~}
          & \includegraphics[height=22pt]{pdf/tuples/8tuple_0_page1.pdf} & \raisebox{10pt}{428,717,762}\\
   \hline
   \raisebox{10pt}{\textsf{8F}}\raisebox{28pt}{~}
          & \includegraphics[height=22pt]{pdf/tuples/8tuple_6_page1.pdf}~
            \includegraphics[height=22pt]{pdf/tuples/8tuple_6_page2.pdf}~
            \includegraphics[height=22pt]{pdf/tuples/8tuple_6_page3.pdf} & \raisebox{10pt}{1,286,153,286}\\
   \hline
   \raisebox{10pt}{\textsf{9F}}\raisebox{28pt}{~}
          & \includegraphics[height=22pt]{pdf/tuples/9tuple_0_page1.pdf} & \raisebox{10pt}{4,715,895,382}\\
   \hline
  \end{tabular}
\end{table}

\subsection{Nタプルネットワークの学習方法}

Nタプルネットワークの重みは,afterstate 間の評価値の差に基づくTD学習法の改良手法によって調整した.
本研究で用いるNタプルネットワークの学習では,以下の技術を用いた.
\begin{description}
  % \item [対称性サンプリング] 各タプルについて,鏡面,回転対称用いてを1つの盤面から,8つの対称位置からサンプリングする. これにより,少ないタプル数で盤面全体から特徴を抽出することができる.
  \item [Multistaging] ゲームの進行に応じて重みを参照するテーブルを切り替える.本研究では,2ステージとし,512 のタイルができる前後でステージを分けた.
  \item [Temporal coherence 学習(TC学習)] TC学習は学習率自動調整機能を備えたTD学習で,Ja\'{s}kowski~\cite{Jask17}が始めて2048に導入した.TC学習では,更新しようとする重みごとに,それまでの学習ステップの更新量の総和を,更新量の絶対値の総和で割った値を学習率とする.本研究の実装では,次の Optimistic initialization の効果がある程度残るように,学習率の最大を 0.5 でクリッピングする変更を加えた.
  \item [Optimistic initialization] 学習段階での探索を広く行うために,重みを(ゼロではなく)大きな値で初期化する.本研究で用いたNタプルの学習では,すべての afterstate の初期値を $\mathit{OI}=0$,$\mathit{OI}=1200$,$\mathit{OI}=5400$ の3通りとした.
\end{description}

それぞれのNタプルニューラルネットワークに対して,$5\times 10^8$ 局面分のデータで学習を行った.著者らの先行研究~\cite{TeKM23}において,この学習量は学習が収束するのに十分であった.

\subsection{Nタプルネットワークを評価関数とするプレイヤ}

本研究で用いるプレイヤは,前節の方法で重みを調整したNタプルネットワークを評価関数とし,Expectimax探索により手を選択する.
Expectimax探索の実装については,著者らの先行研究~\cite{Terauchi24} で用いたものをそのまま利用した.

以上より本研究で用いる各プレイヤは,Nタプルネットワークの組合せ(\textsf{1F}, \textsf{2F}, \ldots, \textsf{9F}, \textsf{3M}, \ldots, \textsf{8M}),学習における Optimistic initialization の初期値($\mathit{OI}\in \{0, 1200, 5400\}$),探索の深さ(Greedy,$d\in \{2, 3, \ldots, 6\}$) の3つにより決定される.(先読みが1手の場合は,慣習に従い Greedy と表記する.)

% \newpage
% \section{Expectimax探索}
% \label{sec:expectimax}
% Expectimax探索は,確率的一人ゲームにおける標準的な探索手法である.
% ミニ2048のゲームの進行は,state におけるプレイヤの選択と,afterstate における新規タイルの出現が交互に起こる.
% したがって,ミニ2048のゲーム木は,根が現在の state に対応し,根から葉への各パス上に,afterstate に対応するノード(chance ノード)と state に対応するノード(max ノード)が交互に現れる.本研究では,ミニ2048のゲーム木の高さ(探索の深さ)を,各パス上の afterstate に対応するノードの数と定める.例えば,高さ2 のミニ2048のゲーム木は,根,afterstate に対応するノードの層,state に対応するノードの層,afterstate に対応するノードの層,の合計4層からなる(図~\ref{result.Expectimax}).

% Expectimax探索では,ゲーム木の各ノードに対して次のように再帰的に計算を行う.
% \begin{itemize}
%  \item maxノードでは,子要素の値のうちの最大値を計算する.
%  \item chanceノードでは,子要素の値を,その出現確率を用いた重み付き平均を計算する.
% \end{itemize}
% Expectimax探索プレイヤは,Expectimax探索によって得られた子ノードのうち,評価値の最も大きなものを選択する.

% 図\ref{result.Expectimax}に深さ2のExpectimax探索の例を示す.

% \begin{figure}[t]
%   \centering
%   \includegraphics[width=.99\linewidth]{pdf/expectimax.pdf}
%   \caption{深さ 2 のExpectimax 探索の例}
%   \label{result.Expectimax} 
% \end{figure}

% ミニ2048のゲーム木では,特に,同じ afterstate が複数出現する.
% そのような同じ afterstate をまとめる(合流)工夫を実装した.ただし,合流を考慮しない Expectimax と結果が一致するよう,同じ afterstate であってもゲーム木中の深さが異なる場合には別のものとして扱った.この工夫により,特に深い探索において大幅な高速化が実現された.
% 本研究では\cite{Terauchi24}で実装したExpectimax探索を用いた.
% \newpage
\section{実験}

前節で説明した方法により,15種類のNタプルネットワークのそれぞれについて,Optimistic initialization の初期値 $\mathit{OI}$ を3通り変えて学習を行った.
ランダム性の影響を抑えるため,各条件について乱数のシードを変えて10回の学習を行い,10個のNタプルネットワークを得た.
次に,各Nタプルネットワークに対し,GreedyプレイおよびExpectimax探索(深さ2〜6)により1000ゲームのプレイを行い,それらの平均スコアを求めた.
本節のグラフにおいて,10個のNタプルネットワークの平均を点や線で示し,それらの標準偏差をエラーバー等で示す.

% 本研究では,ミニ2048におけるNタプルネットワークの学習性能を多角的に分析するため,
% 構造の異なる複数のプレイヤを構築し,GreedyおよびExpectimax探索による評価を行った.

% \ref{tuples}で示した15種類のプレイヤを用いて評価を行った.

% さらに,Optimistic Initialization(OI)の影響を調べるため,
% 各プレイヤについてOIの初期値を0,1200,5400に設定し,それぞれ学習を実施した.
% すべてのプレイヤは $5 \times 10^8$ 手分の行動に基づいて学習を行い,
% 乱数シードを変えて10体ずつ学習させた.結果として,
% $15$タプル構成$\times 3$回(OIの初期値)$\times 10$回(シード)で,計$450$体のプレイヤが作成された.

% これらのプレイヤに対して,1000ゲームプレイを行い,
% seed違いの結果をまとめた10000ゲームのログを用いて解析を行なった.

\begin{figure}[t]
    \centering
    \begin{subfigure}[b]{\linewidth}
        \centering
        \includegraphics[width=\linewidth]{pdf/parameter_performance_plots/params_performance_OI0_EXP1.pdf}
        \caption{OI=0}
        \label{fig:score_vs_tuple_OI0}
    \end{subfigure}

    \vspace{1em}
    \begin{subfigure}[b]{\linewidth}
        \centering
        \includegraphics[width=\linewidth]{pdf/parameter_performance_plots/params_performance_OI1200_EXP1.pdf}
        \caption{OI=1200}
        \label{fig:score_vs_tuple_OI1200}
    \end{subfigure}

    \vspace{1em}
    \begin{subfigure}[b]{\linewidth}
        \centering
        \includegraphics[width=\linewidth]{pdf/parameter_performance_plots/params_performance_OI5400_EXP1.pdf}
        \caption{OI=5400}
        \label{fig:score_vs_tuple_OI5400}
    \end{subfigure}

    \caption{Greedyプレイにおいて,パラメータ数と平均スコアの関係}
    \label{fig:score_vs_tuple_all}
\end{figure}

\begin{figure}[t]
    \centering
    \begin{subfigure}[b]{\linewidth}
        \centering
        \includegraphics[width=\linewidth]{pdf/parameter_performance_plots/params_performance_OI0_EXP6.pdf}
        \caption{OI=0}
        \label{fig:score_vs_tuple_OI0_EXP6}
    \end{subfigure}

    \vspace{1em}
    \begin{subfigure}[b]{\linewidth}
        \centering
        \includegraphics[width=\linewidth]{pdf/parameter_performance_plots/params_performance_OI1200_EXP6.pdf}
        \caption{OI=1200}
        \label{fig:score_vs_tuple_OI1200_EXP6}
    \end{subfigure}

    \vspace{1em}
    \begin{subfigure}[b]{\linewidth}
        \centering
        \includegraphics[width=\linewidth]{pdf/parameter_performance_plots/params_performance_OI5400_EXP6.pdf}
        \caption{OI=5400}
        \label{fig:score_vs_tuple_OI5400_EXP6}
    \end{subfigure}

    \caption{Expectimax深さ6において,パラメータ数と平均スコアの関係}
    \label{fig:score_vs_tuple_all_EXP6}
\end{figure}

\subsection{スコアとパラメータ数の関係}
第1節で示した RQ1 について考察するため,Greedy プレイのスコアを,Optimistic initialization の初期値($\mathit{OI}$)ごとにプロットしたものが図\ref{fig:score_vs_tuple_OI0}から図\ref{fig:score_vs_tuple_OI5400}である.これらのグラフは,横軸にパラメータ数の対数をとり,縦軸にスコアの平均値と標準偏差をプロットしている.また,それぞれのグラフの点に対し,パラメータ数の対数とスコアの関係を二次関数でフィッティングして得られる近似曲線も描いている.

これらのグラフから,いずれのグラフもおよそ放物線を描いていることが分かる.
とくに,\textsf{5M}から\textsf{6F}の区間に放物線の頂点が位置することが確認できた.
また,$\mathit{OI}=0$の場合(図\ref{fig:score_vs_tuple_OI0}),3種類の初期値の中で標準偏差が大きいものが目立っている.このことは,Optimistic initializationを行わない学習では,学習の幅広さが足りず,安定的に良い結果が得られないことを意味する.
$\mathit{OI}=1200$ と $\mathit{OI}=5400$の場合(図\ref{fig:score_vs_tuple_OI1200},図\ref{fig:score_vs_tuple_OI5400}),スコアのばらつきは小さい.
また,より多くのパラメータ数のところまでスコアの向上が見られる(すなわち,放物線の頂点が右に移動する).
\textsf{5F}よりもパラメータ数の少ないNタプルネットワークでは,平均スコアは初期値にそれほど依存していない.一方,それよりも多くのパラメータを持つNタプルネットワークでは,初期値が $\mathit{OI}=1200$ の場合に最も良い結果が得られた.

次に,RQ1 と RQ3 について考察するため,深さ6のExpectimax探索を行った場合の,Nタプルネットワークのパラメータ数と平均スコアを関係を図\ref{fig:score_vs_tuple_OI0_EXP6}から図\ref{fig:score_vs_tuple_OI5400_EXP6}に示す.
各グラフの近似曲線から,いずれのグラフもおよそ放物線を描いていること,いずれの平均スコアもGreedyプレイのスコアよりも高いことが確認できた.

$\mathit{OI}=0$ の場合,探索を行ってもスコアのばらつきはそれほど小さくならず,特に Full のものについてばらつきが大きい傾向が見られる.
これはパラメータ数が多い方が評価値の修正が起こり難く,局所最適解から抜け出しにくいのではないかと考えられる.

$\mathit{OI}=1200$ の場合,\textsf{5M}から\textsf{7F}まで同程度の平均スコアを達成しており,放物線の上昇と下降の傾きが小さい.
これは,強いプレイヤが達成しうるスコアの上限に近づいていて向上の余地が小さいことと,弱いプレイヤが探索によってスコアを上昇させられることを示唆する.
$\mathit{OI}=5400$ の場合には,スコアのばらつきは小さいものの,放物線の形やスコアの最大は $\mathit{OI}=0$ の場合のそれらとあまり変わらなかった.


\subsection{学習の進み方の比較}

RQ2 について考察するため,\textsf{5M}と\textsf{7M}を例にとり,学習過程のスコアの推移を確認した.図\ref{fig:learning_progress_comparison}は,横軸に学習ステップ数を,縦軸にスコアをとり,学習ステップ10000ごとに学習エピソードの平均をプロットしたものである.

Optimistic initializationの初期値を $\mathit{OI}=0$ と設定した場合,\textsf{5M}では学習が進むにつれてスコアが上昇しているが,\textsf{7F}ではスコアが上昇していない.
これは\textsf{7F}が局所最適解にハマっていることを示唆する.

初期値を$\mathit{OI}=1200$と設定した場合,\textsf{5M}と\textsf{7F}の両方でスコアの上昇が $\mathit{OI}=0$ の場合よりも緩やかになった.
詳しく見ると,\textsf{5M}では,約2800点を達成したあたりで途中一度停滞しており,その後再びスコアの上昇に転じている.
表\ref{pp-specific}より,約2800点というのは256タイルが完成してから512タイルが完成するまでの間であることが分かる.
この間の盤面では空きタイルが多くあるため,学習に出現する盤面の種類が多いことが停滞の原因ではないかと考えている.
また,\textsf{7F}において初期値を $\mathit{OI}=1200$ と設定した場合,序盤のスコアの上昇が緩やかになっているが,これも同じ原因ではないかと考える.

初期値を $\mathit{OI}=5400$ と設定した場合,\textsf{5M}と\textsf{7F}の両方でスコアの上昇が緩やかになり,途中で停滞が発生している.
\textsf{5M}では停滞を乗り越えて大きくスコアを上昇させることに成功しているが,\textsf{7F}では停滞が長引いてしまい結果として学習不足であることが判明した.

\begin{figure}[t]
    \centering
    \begin{subfigure}[b]{\linewidth}
        \centering
        \includegraphics[width=\linewidth]{pdf/learning_progress_plots/learning_progress_NT5M_tuple298_combined.pdf}
        \caption{5Mの学習過程のスコアの変化}
        \label{fig:learning_progress_NT5M}
    \end{subfigure}

    \vspace{1em}
    \begin{subfigure}[b]{\linewidth}
        \centering
        \includegraphics[width=\linewidth]{pdf/learning_progress_plots/learning_progress_NT7F_tuple0_combined.pdf}
        \caption{7Fの学習過程のスコアの変化}
        \label{fig:learning_progress_NT7F}
    \end{subfigure}

    \caption{5Mと7Fの学習推移の比較}
    \label{fig:learning_progress_comparison}
\end{figure}

\subsection{パラメータ数の増加によるプレイヤの挙動の変化}
% ここからRQ1についての考察を深めるために,
OIの初期値1200のスコアが同等でパラメータ数が違う4Fと8M,5Mと7Fのプレイヤをパーフェクトプレイヤを用いて詳細な比較を行い,
パラメータ数の増加がプレイヤの挙動にどのような影響を与えるのかについて詳しく調べて行く.
指標としては,正確度,絶対誤差,生存率を用いる.
\begin{itemize}
    \item 正確度:パーフェクトプレイヤの選択した手とプレイヤの選択した手の一致率
    \item 絶対誤差:パーフェクトプレイヤの選択した手とプレイヤの選択した手のパーフェクトプレイヤ評価値の差
    \item 生存率:あるprogressにおいて,プレイヤが生存している確率
\end{itemize}

\begin{figure}[t]
\centering
\begin{subfigure}[b]{0.8\linewidth}
    \includegraphics[width=\linewidth]{pdf/compare/NT4F_and_NT8M/accuracy.pdf}
    \caption{正確度}
    \label{fig:NT4F_and_NT8M_accuracy}
\end{subfigure}
\begin{subfigure}[b]{0.8\linewidth}
    \includegraphics[width=\linewidth]{pdf/compare/NT4F_and_NT8M/error_abs.pdf}
    \caption{絶対誤差}
    \label{fig:NT4F_and_NT8M_error_abs}
\end{subfigure}
\begin{subfigure}[b]{0.8\linewidth}
    \includegraphics[width=\linewidth]{pdf/compare/NT4F_and_NT8M/survival.pdf}
    \caption{生存率}
    \label{fig:NT4F_and_NT8M_survival}
\end{subfigure}
\caption{4Fと8Mの比較結果}
\label{fig:NT4F_and_NT8M_results}
\end{figure}

まず初めに図\ref{fig:NT4F_and_NT8M_results}は,4Fと8MのプレイヤのGreedyプレイの比較を示している.
図\ref{fig:NT4F_and_NT8M_accuracy}は,を見ると正確度のグラフはどちらも同じような形をしているが,8Mの方が上下に大きく変動していることが分かる.
図\ref{fig:NT4F_and_NT8M_error_abs}を見ると4Fの方が8M多くのprogressで絶対誤差が小さいことが分かる,これはprogress260を超えた辺りから顕著に現れている.
progress260辺りは512のタイルが完成し以前と似ている盤面になるのだが,8Mタプルサイズが大きくなることで汎化性能が落ちて,
序盤の盤面と全く別物として学習してしまい,512タイルが完成した後の盤面に対して正確度が下がり,絶対誤差も大きくなってるのではないかと考えられる.
これは図\ref{fig:NT4F_and_NT8M_survival}の生存率にも表れていて生存率ではミスをした後のprogress300を超えた辺りから顕著に生存率が8Mの生存率が下がっている.

\begin{figure}[t]
\centering
\begin{subfigure}[b]{0.8\linewidth}
    \includegraphics[width=\linewidth]{pdf/compare/NT5M_and_NT7F/accuracy.pdf}
    \caption{正確度}
    \label{fig:NT5M_and_NT7F_accuracy}
\end{subfigure}
\begin{subfigure}[b]{0.8\linewidth}
    \includegraphics[width=\linewidth]{pdf/compare/NT5M_and_NT7F/error_abs.pdf}
    \caption{絶対誤差}
    \label{fig:NT5M_and_NT7F_error_abs}
\end{subfigure}
\begin{subfigure}[b]{0.8\linewidth}
    \includegraphics[width=\linewidth]{pdf/compare/NT5M_and_NT7F/survival.pdf}
    \caption{生存率}
    \label{fig:NT5M_and_NT7F_survival}
\end{subfigure}
\caption{5Mと7Fの比較結果}
\label{fig:NT5M_and_NT7F_results}
\end{figure}

次に図\ref{fig:NT5M_and_NT7F_results}は,5Mと7FのプレイヤのGreedyプレイの比較を示している.
図\ref{fig:NT5M_and_NT7F_accuracy}を見ると形は似ているが,両方の正確度が下がる場面で7Fの方が正確度が下がっているのが分かる.
図\ref{fig:NT5M_and_NT7F_error_abs}を見ると,どちらもほぼ同じ形になっていて,正確度ほど差のないグラフになっている
これは7Fは間違えても問題ない手を選んでいるだけで学習自体は十分に成功していることがわかる.
図\ref{fig:NT5M_and_NT7F_survival}を見るとどちらも上下を入れ替わりながら似たような形になっていることが分かる.
    
\begin{figure}[t]
\centering
\begin{subfigure}[b]{0.8\linewidth}
    \includegraphics[width=\linewidth]{pdf/compare/EXP6_NT4F_and_NT8M/accuracy.pdf}
    \caption{正確度}
    \label{fig:EXP6_NT4F_and_NT8M_accuracy}
\end{subfigure}
\begin{subfigure}[b]{0.8\linewidth}
    \includegraphics[width=\linewidth]{pdf/compare/EXP6_NT4F_and_NT8M/error_abs.pdf}
    \caption{絶対誤差}
    \label{fig:EXP6_NT4F_and_NT8M_error_abs}
\end{subfigure}
\begin{subfigure}[b]{0.8\linewidth}
    \includegraphics[width=\linewidth]{pdf/compare/EXP6_NT4F_and_NT8M/survival.pdf}
    \caption{生存率}
    \label{fig:EXP6_NT4F_and_NT8M_survival}
\end{subfigure}
\caption{4Fと8Mの比較結果(深さ6)}
\label{fig:EXP6_NT4FとNT8M_results}
\end{figure}
    

\begin{figure}[t]
\centering
\begin{subfigure}[b]{0.8\linewidth}
    \includegraphics[width=\linewidth]{pdf/compare/EXP6_NT5M_and_NT7F/accuracy.pdf}
    \caption{正確度}
    \label{fig:EXP6_NT5M_and_NT7F_accuracy}
\end{subfigure}
\begin{subfigure}[b]{0.8\linewidth}
    \includegraphics[width=\linewidth]{pdf/compare/EXP6_NT5M_and_NT7F/error_abs.pdf}
    \caption{絶対誤差}
    \label{fig:EXP6_NT5M_and_NT7F_error_abs}
\end{subfigure}
\begin{subfigure}[b]{0.8\linewidth}
    \includegraphics[width=\linewidth]{pdf/compare/EXP6_NT5M_and_NT7F/survival.pdf}
    \caption{生存率}
    \label{fig:EXP6_NT5M_and_NT7F_survival}
\end{subfigure}
\caption{5Mと7Fの比較結果(深さ6)}
\label{fig:EXP6_NT5M_and_NT7F_results}
\end{figure}

図\ref{fig:EXP6_NT4FとNT8M_results}と図\ref{fig:EXP6_NT5M_and_NT7F_results}は,
図\ref{fig:NT4F_and_NT8M_results}と図\ref{fig:NT5M_and_NT7F_results}のExpectimax探索深さ6を組み合わせたプレイヤである.
それぞれのスコアとしては向上していてそれは図\ref{fig:EXP6_NT4F_and_NT8M_error_abs}と図\ref{fig:EXP6_NT5M_and_NT7F_error_abs}
の絶対誤差と図\ref{fig:EXP6_NT4F_and_NT8M_survival}と図\ref{fig:EXP6_NT5M_and_NT7F_survival}の生存率に現れている.
図\ref{fig:EXP6_NT4F_and_NT8M_accuracy}と図\ref{fig:EXP6_NT5M_and_NT7F_accuracy}の正確度はどちらも形はGreedyと似たような形になってる.
\section{考察}
\label{sec:consideration}
ミニ2048においてNタプルネットワークのパラメータ数とスコアの関係について詳細な分析を行った結果
パラメータ数のlogとスコアの関係は方物線的であることが確認された.
2048においても同様の傾向だとすると1から16までのタプルのなかでミニ2048の5か6に匹敵するのは8タプル,9タプル,10タプルあたりであると考えられる.
本研究では図\ref{fig:NT4F_and_NT8M_results}と図\ref{fig:NT5M_and_NT7F_results},図\ref{fig:EXP6_NT4FとNT8M_results}と図\ref{fig:EXP6_NT5M_and_NT7F_results}
の比較でスコアが同程度の場合のパラメータ数の差がある場合の挙動について分析をしたが,グラフの形こそ同じようになるものの
ある部分が致命的に弱いというようなことはなかったので,2048で8タプル,9タプル,10タプルを用いた場合の学習性能は既存研究の6タプルを上回るようなスコアを期待できる.
\section{まとめ}
本研究では,ミニ2048におけるNタプルネットワークのタプルサイズおよびOptimistic Initialization(OI)の初期値が,
プレイヤの学習性能および探索によるスコアに与える影響について詳細に分析した.
実験の結果,以下の重要な知見が得られた:
\begin{itemize}
\item パラメータ数の増加とスコアの関係は方物線的であり,ミニ2048では5か6を頂点とすることが確認された.
\item OIの初期値は0ではスコアのばらつきが大きく,局所最適解にハマるのか,学習の安定性に影響を与えることが確認された.
しかし大きな値を設定しすぎると学習が終わらない可能性も示された.
\item OIの初期値を適切に設定すると,スコアの向上が続くパラメータ数が増加する傾向が確認できた.
\item 探索を組み合わせることでどのパラメータ数でもスコアが向上することが確認された.
\item 探索を実装した場合でもOIの値と平均スコアのばらつきの関係は変わらなかった.
\end{itemize}

以上の結果から
ミニ2048においてパラメータ数とNタプルネットワークのサイズがスコアに与える影響とOptimistic Initialization(OI)の初期値がスコアに与える影響を明らかにした.
今後の課題としては,これらの結果を踏まえて,2048で1から9までのタプルを用いた場合の学習性能を評価することと,
Multistagingによってパラメータ数を変化させた場合の学習性能を評価することが挙げられる.
それによって得られたタプルをExpectimax探索に組み込むことで,より高いスコアを達成することが挙げられる.
% 5,6,7+別で8 1200
% 横軸にパラメータ数,縦軸にスコアのやつを入れる
% 2048で8タプルか9タプルのラインでいいんじゃないか?
% 全てのタプルのOI1200の平均点
% タプルの表で横にずらしたやつは並べる
% 増やす時は対応してると嬉しい,追加分は右に
% F (Full) vs M (Manual)
% Expectimax探索の深さ6と学習のスコアの曲線が欲しい
% 詳細な比較をタプル数の違いで比較する

\begin{acknowledgment}
    本研究はJSPS科研費 JP23K11383 の助成を受けたものである.
    % 本研究を進めるにあたり,高知工科大学 情報学群の松崎公紀教授には,貴重なご指導と多くの助言を賜りました.心より感謝申し上げます.
    % また,本論文の副査をお引き受けいただいた Wei Ting Han 教授,竹内聖悟 講師にも,深く感謝申し上げます.
    % さらに,研究室の同期である金子氏とは,研究環境の整備を共同で行い,多くの議論を重ねました.その協力に心より感謝いたします.
\end{acknowledgment}

% BibTeX を使用する場合 %%%%%%%%%%%%%%%%%%%%%%%%%%%%%%%%%
\def\newblock{}

% \input{ref.bib}
\bibliographystyle{ipsjsort}
\bibliography{ref}



\end{document}
