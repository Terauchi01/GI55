\section{考察}
\label{sec:consideration}
ミニ2048においてNタプルネットワークのパラメータ数とスコアの関係について詳細な分析を行った結果
パラメータ数のlogとスコアの関係は方物線的であることが確認された.
2048においても同様の傾向だとすると1から16までのタプルのなかでミニ2048の5か6に匹敵するのは8タプル,9タプル,10タプルあたりであると考えられる.
本研究では図\ref{fig:NT4F_and_NT8M_results}と図\ref{fig:NT5M_and_NT7F_results},図\ref{fig:EXP6_NT4FとNT8M_results}と図\ref{fig:EXP6_NT5M_and_NT7F_results}
の比較でスコアが同程度の場合のパラメータ数の差がある場合の挙動について分析をしたが,グラフの形こそ同じようになるものの
ある部分が致命的に弱いというようなことはなかったので,2048で8タプル,9タプル,10タプルを用いた場合の学習性能は既存研究の6タプルを上回るようなスコアを期待できる.