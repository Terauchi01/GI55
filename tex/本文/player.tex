\section{本研究で用いるプレイヤ}
\subsection{Nタプルネットワーク評価関数}
\label{sec:Ntuple}

2048における最も成功したプレイヤの多くは,Nタプルネットワークに基づく評価関数を強化学習によってチューニングするアプローチを採用している~\cite{SzJa14}.
Gueiらの最新のプレイヤ~\cite{GuCW22}も,Matsuzaki~\cite{Mats16}が提案したタプルの組合せをベースに,Expectimax探索やMultistaging\cite{YWHC16},Optimistic Initialization,Tile Downgrading\cite{GuCW22}などの改良を加えることで高い性能を達成している.

本研究では,こうした知見を踏まえ,ミニ2048において1タプルから9タプルまでのNタプルネットワークを構築可能な全てのタプル列挙を行い,
\begin{itemize}
  \item M(Manual): 人間の知見に基づいて、効果的と考えられるタイル位置のみを選択して構成したタプル集合。3Mなどの"M"はこれを表す
  \item F(Full): 指定したサイズのタプルについて、盤面上で連続するタイルの全ての組み合わせを使用したタプル集合。1Fなどの"F"はこれを表す
\end{itemize}

この2つの設計方針に基づき、タプル数(N)を1から9まで変化させてネットワークを構築した。
ここで、プレイヤ名の数字はタプル数を、末尾のMとFはそれぞれManualとFullの設計方針を表している。

この結果,最大で18種類のNタプルネットワークが得られたが,1タプル,2タプル,9タプルにおいては両方の設計が一致したため,プレイヤの種類としては15通りとなった.
これらのプレイヤを用いて,Nタプルネットワークの大きさ(タプル数)と学習性能の関係を実験的に評価した.

表\ref{tuples}に,本研究で使用したタプルの構造を示す.
また,ミニ2048の盤面の持つ対称性(回転・反転)**を活用し,1つのタプルに対して対称な8通りの位置からのサンプリングを行うことで,学習に必要なタプル数の削減を図っている.

\begin{table}[t]
  \caption{タプルサイズと形状の一覧}
  \label{tuples}
  \centering\begin{tabular}{llr}
   \hline
   \hline
   タプル名 & \hspace{20pt}タプルの組合せ & パラメータ数\\
   \hline
   \raisebox{10pt}{1F}\raisebox{28pt}{~}
          & \includegraphics[height=22pt]{pdf/tuples/1tuple_6_page1.pdf}~
            \includegraphics[height=22pt]{pdf/tuples/1tuple_6_page2.pdf}~
            \includegraphics[height=22pt]{pdf/tuples/1tuple_6_page3.pdf} & 66\\
   \hline
   \raisebox{10pt}{2F}\raisebox{28pt}{~}
          & \includegraphics[height=22pt]{pdf/tuples/2tuple_12_page1.pdf}~
            \includegraphics[height=22pt]{pdf/tuples/2tuple_12_page2.pdf}& 484\\
   \hline
   \raisebox{10pt}{3M}\raisebox{28pt}{~}
          & \includegraphics[height=22pt]{pdf/tuples/3tuple_144_page1.pdf}~
            \includegraphics[height=22pt]{pdf/tuples/3tuple_144_page3.pdf}~
            \includegraphics[height=22pt]{pdf/tuples/3tuple_144_page2.pdf}& 7,986\\
   \hline
   \raisebox{10pt}{3F}\raisebox{28pt}{~}
          & \includegraphics[height=22pt]{pdf/tuples/3tuple_2673_page1.pdf}~
            \includegraphics[height=22pt]{pdf/tuples/3tuple_2673_page5.pdf}~
            \includegraphics[height=22pt]{pdf/tuples/3tuple_2673_page2.pdf}~
            \includegraphics[height=22pt]{pdf/tuples/3tuple_2673_page4.pdf}~
            \includegraphics[height=22pt]{pdf/tuples/3tuple_2673_page3.pdf}& 13,310\\
   \hline
   \raisebox{10pt}{4M}\raisebox{28pt}{~}
          & \includegraphics[height=22pt]{pdf/tuples/4tuple_301_page1.pdf}~
            \includegraphics[height=22pt]{pdf/tuples/4tuple_301_page3.pdf}~
            \includegraphics[height=22pt]{pdf/tuples/4tuple_301_page2.pdf}& 87,846\\
   \hline
   \raisebox{10pt}{4F}\raisebox{28pt}{~}
          & \includegraphics[height=22pt]{pdf/tuples/4tuple_44755_page1.pdf}~
            \includegraphics[height=22pt]{pdf/tuples/4tuple_44755_page5.pdf}~
            \includegraphics[height=22pt]{pdf/tuples/4tuple_44755_page3.pdf}& \multirow{2}{*}{175,692}\\
          & \includegraphics[height=22pt]{pdf/tuples/4tuple_44755_page6.pdf}~
            \includegraphics[height=22pt]{pdf/tuples/4tuple_44755_page2.pdf}~
            \includegraphics[height=22pt]{pdf/tuples/4tuple_44755_page4.pdf}\\
   \hline
   \raisebox{10pt}{5M}\raisebox{28pt}{~}
          & \includegraphics[height=22pt]{pdf/tuples/5tuple_298_page1.pdf}~
            \includegraphics[height=22pt]{pdf/tuples/5tuple_298_page2.pdf}~
            \includegraphics[height=22pt]{pdf/tuples/5tuple_298_page3.pdf} & 966306\\
   \hline
   \raisebox{10pt}{5F}\raisebox{28pt}{~}
          & \includegraphics[height=22pt]{pdf/tuples/5tuple_896673_page1.pdf}~
            \includegraphics[height=22pt]{pdf/tuples/5tuple_896673_page3.pdf}~
            \includegraphics[height=22pt]{pdf/tuples/5tuple_896673_page4.pdf}~
            \includegraphics[height=22pt]{pdf/tuples/5tuple_896673_page2.pdf}~
            \includegraphics[height=22pt]{pdf/tuples/5tuple_896673_page5.pdf}& \multirow{2}{*}{2,898,918}\\
          & \includegraphics[height=22pt]{pdf/tuples/5tuple_896673_page6.pdf}~
            \includegraphics[height=22pt]{pdf/tuples/5tuple_896673_page7.pdf}~
            \includegraphics[height=22pt]{pdf/tuples/5tuple_896673_page8.pdf}~
            \includegraphics[height=22pt]{pdf/tuples/5tuple_896673_page9.pdf}\\
   \hline
   \raisebox{10pt}{6M}\raisebox{28pt}{~}
          & \includegraphics[height=22pt]{pdf/tuples/6tuple_16_page1.pdf}~
            \includegraphics[height=22pt]{pdf/tuples/6tuple_16_page2.pdf} & 7086244\\
   \hline
   \raisebox{10pt}{6F}\raisebox{28pt}{~}
          & \includegraphics[height=22pt]{pdf/tuples/6tuple_26835_page1.pdf}~
            \includegraphics[height=22pt]{pdf/tuples/6tuple_26835_page2.pdf}~
            \includegraphics[height=22pt]{pdf/tuples/6tuple_26835_page3.pdf}~
            \includegraphics[height=22pt]{pdf/tuples/6tuple_26835_page4.pdf}& \multirow{2}{*}{28,344,976}\\
          & \includegraphics[height=22pt]{pdf/tuples/6tuple_26835_page5.pdf}~
            \includegraphics[height=22pt]{pdf/tuples/6tuple_26835_page6.pdf}~
            \includegraphics[height=22pt]{pdf/tuples/6tuple_26835_page7.pdf}~
            \includegraphics[height=22pt]{pdf/tuples/6tuple_26835_page8.pdf}\\
   \hline
   \raisebox{10pt}{7M}\raisebox{28pt}{~}
          & \includegraphics[height=22pt]{pdf/tuples/7tuple_0_page1.pdf} & 38,974,342\\
   \hline
   \raisebox{10pt}{7F}\raisebox{28pt}{~}
          & \includegraphics[height=22pt]{pdf/tuples/7tuple_248_page1.pdf}~
            \includegraphics[height=22pt]{pdf/tuples/7tuple_248_page2.pdf}~
            \includegraphics[height=22pt]{pdf/tuples/7tuple_248_page3.pdf}~
            \includegraphics[height=22pt]{pdf/tuples/7tuple_248_page4.pdf}& \multirow{2}{*}{272,820,394}\\
          & \includegraphics[height=22pt]{pdf/tuples/7tuple_248_page5.pdf}~
            \includegraphics[height=22pt]{pdf/tuples/7tuple_248_page6.pdf}~
            \includegraphics[height=22pt]{pdf/tuples/7tuple_248_page7.pdf}\\
   \hline
   \raisebox{10pt}{8M}\raisebox{28pt}{~}
          & \includegraphics[height=22pt]{pdf/tuples/8tuple_0_page1.pdf} & 428,717,762\\
   \hline
   \raisebox{10pt}{8F}\raisebox{28pt}{~}
          & \includegraphics[height=22pt]{pdf/tuples/8tuple_6_page1.pdf}~
            \includegraphics[height=22pt]{pdf/tuples/8tuple_6_page2.pdf}~
            \includegraphics[height=22pt]{pdf/tuples/8tuple_6_page3.pdf} & 1,286,153,286\\
   \hline
   \raisebox{10pt}{9F}\raisebox{28pt}{~}
          & \includegraphics[height=22pt]{pdf/tuples/9tuple_0_page1.pdf} & 4,715,895,382\\
   \hline
  \end{tabular}
\end{table}

Nタプルネットワークの重みは,afterstate 間の評価値の差に基づくTD学習法の改良手法によって調整した.
本研究で用いるNタプルネットワークの学習では,以下の技術を用いた.
\begin{description}
  % \item [対称性サンプリング] 各タプルについて,鏡面,回転対称用いてを1つの盤面から,8つの対称位置からサンプリングする. これにより,少ないタプル数で盤面全体から特徴を抽出することができる.
  \item [Multistaging] ゲームの進行に応じて重みを参照するテーブルを切り替える.本研究では,2ステージとし,512 のタイルができる前後でステージを分けた.
  \item [Temporal coherence 学習(TC学習)] TC学習は学習率自動調整機能を備えたTD学習で,Ja\'{s}kowski~\cite{Jask17}が始めて2048に導入した.
  \item [Optimistic initialization (OI)] 学習段階での探索を広く行うために,重みを(ゼロではなく)大きな値で初期化する.本研究で用いたNタプルの学習では,すべての afterstate の初期値が 1200 になるように重みを初期化してある.
\end{description}

それぞれのNタプルニューラルネットワークに対して,$5\times 10^8$ 局面分のデータで学習を行った.いずれのニューラルネットワークも,十分に学習が収束していることを確認してある~\cite{TeKM23}.
% \newpage
% \section{Expectimax探索}
% \label{sec:expectimax}
% Expectimax探索は,確率的一人ゲームにおける標準的な探索手法である.
% ミニ2048のゲームの進行は,state におけるプレイヤの選択と,afterstate における新規タイルの出現が交互に起こる.
% したがって,ミニ2048のゲーム木は,根が現在の state に対応し,根から葉への各パス上に,afterstate に対応するノード(chance ノード)と state に対応するノード(max ノード)が交互に現れる.本研究では,ミニ2048のゲーム木の高さ(探索の深さ)を,各パス上の afterstate に対応するノードの数と定める.例えば,高さ2 のミニ2048のゲーム木は,根,afterstate に対応するノードの層,state に対応するノードの層,afterstate に対応するノードの層,の合計4層からなる(図~\ref{result.Expectimax}).

% Expectimax探索では,ゲーム木の各ノードに対して次のように再帰的に計算を行う.
% \begin{itemize}
%  \item maxノードでは,子要素の値のうちの最大値を計算する.
%  \item chanceノードでは,子要素の値を,その出現確率を用いた重み付き平均を計算する.
% \end{itemize}
% Expectimax探索プレイヤは,Expectimax探索によって得られた子ノードのうち,評価値の最も大きなものを選択する.

% 図\ref{result.Expectimax}に深さ2のExpectimax探索の例を示す.

% \begin{figure}[t]
%   \centering
%   \includegraphics[width=.99\linewidth]{pdf/expectimax.pdf}
%   \caption{深さ 2 のExpectimax 探索の例}
%   \label{result.Expectimax} 
% \end{figure}

% ミニ2048のゲーム木では,特に,同じ afterstate が複数出現する.
% そのような同じ afterstate をまとめる(合流)工夫を実装した.ただし,合流を考慮しない Expectimax と結果が一致するよう,同じ afterstate であってもゲーム木中の深さが異なる場合には別のものとして扱った.この工夫により,特に深い探索において大幅な高速化が実現された.
% 本研究では\cite{Terauchi24}で実装したExpectimax探索を用いた.
% \newpage